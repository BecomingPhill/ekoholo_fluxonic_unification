\documentclass{article}
\usepackage{amsmath, amssymb, graphicx}

\title{Fluxonic Black Holes and Emergent Event Horizons}
\author{Independent Theoretical Study}
\date{\today}

\begin{document}

\maketitle

\begin{abstract}
This study explores fluxonic black holes as an alternative to General Relativity's gravitational singularities. Using numerical simulations of the nonlinear Klein-Gordon equation, we analyze fluxonic collapse, event horizon formation, and energy radiation. Our results indicate that fluxonic black holes retain energy over time, suggesting a dynamically stable gravitational structure rather than an evaporating singularity. These findings challenge conventional black hole thermodynamics and suggest new quantum gravity mechanisms.
\end{abstract}

\section{Introduction}
Classical black holes, as described by General Relativity, exhibit event horizons and singularities where gravity becomes infinite. However, alternative approaches suggest that gravity might be an emergent phenomenon driven by solitonic interactions. This paper investigates whether fluxonic solitons can replicate gravitational collapse and event horizons without singularities.

\section{Mathematical Framework}
We model fluxonic black hole formation using the nonlinear Klein-Gordon equation:
\begin{equation}
    \frac{\partial^2 \phi}{\partial t^2} - \frac{\partial^2 \phi}{\partial x^2} + m^2 \phi + g \phi^3 = 0,
\end{equation}
where $\phi(x,t)$ represents the fluxonic field, $m$ is a mass-like parameter, and $g$ governs nonlinear interactions.

The fluxonic collapse is initiated with an exponentially dense field:
\begin{equation}
    \phi(x,0) = e^{-x^2}, \quad \frac{\partial \phi}{\partial t} \Big|_{t=0} = 0.
\end{equation}

\section{Numerical Simulation and Results}
The collapse simulation revealed a reduction in escape velocity over time:
\begin{align*}
    \text{Initial Escape Velocity} &= 0.86, \\
    \text{Final Escape Velocity} &= 0.60, \\
    \text{Reduction} &= 30\%.
\end{align*}

This suggests that fluxonic structures trap energy, mimicking an event horizon. However, the total energy of the system increased over time rather than decreasing:
\begin{align*}
    \text{Initial Energy} &= 0.84, \\
    \text{Final Energy} &= 1.04, \\
    \text{Unexpected Increase} &= 23\%.
\end{align*}

\section{Discussion and Implications}
1. \textbf{Fluxonic Event Horizons:} The reduction in escape velocity suggests that fluxonic collapse mimics gravitational horizons.
2. \textbf{Energy Retention Instead of Loss:} Unlike classical Hawking radiation, fluxonic structures maintain or gain energy, potentially explaining stable dark matter structures.
3. \textbf{No Singularity Required:} Fluxonic black holes may store energy in nonlinear solitonic interactions, eliminating singularities and the information paradox.

\section{Conclusion}
Our results indicate that fluxonic black holes provide a viable alternative to classical singularities, supporting the hypothesis that gravity emerges from solitonic interactions. Future work should investigate their gravitational wave signatures to compare with LIGO observations.

\end{document}

