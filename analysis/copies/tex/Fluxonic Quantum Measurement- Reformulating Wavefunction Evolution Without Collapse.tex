\documentclass{article}
\usepackage{amsmath, amssymb, graphicx, listings}

\title{Fluxonic Quantum Measurement: Reformulating Wavefunction Evolution Without Collapse}
\author{Tshuutheni Emvula and Independent Theoretical Study}
\date{\today}

\begin{document}

\maketitle

\begin{abstract}
This paper develops a fluxonic framework for quantum measurement, proposing that wavefunction evolution emerges deterministically from structured fluxonic wave interactions rather than requiring probabilistic wavefunction collapse. We derive a fluxonic equation replacing Schrödinger’s formulation, numerically simulate a double-slit experiment, and propose an alternative explanation for quantum superposition, measurement, and entanglement. These results challenge the necessity of wavefunction collapse and suggest a deterministic interpretation of quantum mechanics through fluxonic wave interactions.
\end{abstract}

\section{Introduction}
Quantum mechanics describes the evolution of a system via the Schrödinger equation, yet the act of measurement introduces the problem of wavefunction collapse. Standard interpretations rely on probabilistic measurement rules that lack a clear physical mechanism. Here, we propose that measurement arises naturally from structured fluxonic wave interactions, eliminating the need for collapse and allowing quantum superposition to be maintained within a deterministic fluxonic framework.

\section{Fluxonic Wavefunction Evolution}
We propose a fluxonic alternative to the Schrödinger equation:
\begin{equation}
    i\hbar \frac{\partial \psi}{\partial t} = -\frac{\hbar^2}{2m} \frac{\partial^2 \psi}{\partial x^2} + \alpha \psi,
\end{equation}
which is replaced in the fluxonic framework by:
\begin{equation}
    \frac{\partial^2 \phi}{\partial t^2} - c^2 \frac{\partial^2 \phi}{\partial x^2} + \alpha \phi = 0.
\end{equation}
This equation suggests that quantum evolution is not a probabilistic wavefunction collapse but a structured fluxonic wave evolution governed by deterministic field interactions.

\section{Numerical Simulations of Fluxonic Quantum Measurement}
We performed numerical simulations to analyze fluxonic wavefunction behavior:
- **Fluxonic Double-Slit Experiment:** Instead of wavefunction collapse, measurement arises from deterministic fluxonic wave evolution, preserving superposition as structured fluxonic interactions.
- **Fluxonic Quantum Entanglement:** Information exchange occurs through fluxonic correlations rather than non-local probabilistic collapse.
- **Quantum Decoherence:** Environmental fluxonic interactions cause structured wave stability rather than stochastic decoherence.

\section{Reproducible Code for Fluxonic Quantum Simulations}
\subsection{Fluxonic Double-Slit Experiment}
\begin{lstlisting}[language=Python]
import numpy as np
import matplotlib.pyplot as plt

# Define spatial and temporal grid
Nx = 300  # Number of spatial points
Nt = 200  # Number of time steps
L = 10.0  # Spatial domain size
dx = L / Nx  # Spatial step size
dt = 0.01  # Time step

# Define initial fluxonic wave packet (double-slit interference)
x = np.linspace(-L/2, L/2, Nx)
phi = np.exp(-x**2) * np.cos(5 * np.pi * x)  # Initial wave function

# Define slits in space (mimicking double slit experiment)
slit_width = 0.2
barrier = np.ones(Nx)
barrier[np.abs(x - 1.5) < slit_width] = 0  # Left slit
barrier[np.abs(x + 1.5) < slit_width] = 0  # Right slit

# Apply barrier effect to wave function
phi *= barrier

# Initialize previous state
phi_old = np.copy(phi)
phi_new = np.zeros_like(phi)

# Fluxonic parameters
c = 1.0  # Wave propagation speed
alpha = -0.1  # Energy interaction term

# Time evolution loop
for n in range(Nt):
    d2phi_dx2 = (np.roll(phi, -1) - 2 * phi + np.roll(phi, 1)) / dx**2
    phi_new = 2 * phi - phi_old + dt**2 * (c**2 * d2phi_dx2 + alpha * phi)
    phi_old, phi = phi, phi_new

# Plot fluxonic wave interference pattern
plt.figure(figsize=(8, 5))
plt.plot(x, phi, label="Fluxonic Wave Intensity")
plt.xlabel("Position (x)")
plt.ylabel("Wave Amplitude")
plt.title("Fluxonic Double-Slit Interference")
plt.legend()
plt.grid()
plt.show()
\end{lstlisting}

\section{Conclusion}
This work presents a deterministic fluxonic alternative to quantum measurement, suggesting that wavefunction evolution, superposition, and entanglement are structured field effects rather than probabilistic collapses. Future research will focus on experimental tests and implications for unifying quantum mechanics with deterministic wave-based interpretations.

\end{document}

