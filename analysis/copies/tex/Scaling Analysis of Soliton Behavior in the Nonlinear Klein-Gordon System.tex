\documentclass{article}
\usepackage{amsmath, amssymb, graphicx}
\usepackage{hyperref}

\title{Scaling Analysis of Soliton Behavior in the Nonlinear Klein-Gordon System}
\author{}
\date{}

\begin{document}
\maketitle

\section{Introduction}
This document presents an analysis of how soliton properties depend on mass ($m$) and nonlinearity ($g$) in the nonlinear Klein-Gordon system.

\section{Mathematical Framework}
We solve the nonlinear Klein-Gordon equation:
\begin{equation}
\frac{\partial^2 \phi}{\partial t^2} - \frac{\partial^2 \phi}{\partial x^2} + m^2 \phi + g \phi^3 = 0.
\end{equation}
Soliton collisions are analyzed at varying values of $m$ and $g$ to determine their impact on phase shifts and energy conservation.

\section{Results and Observations}
\subsection{Phase Shift Dependence on $m$ and $g$}

table of results:
\begin{center}
\begin{tabular}{|c|c|c|c|c|}
\hline
$m$ & $g$ & Phase Shift (Soliton 1) & Phase Shift (Soliton 2) & Final Energy \\
\hline
0.5 & 0.500 & 0.000 & 2.81 & 20.01 \\
0.5 & 0.875 & -4.82 & 2.21 & 31.09 \\
0.5 & 1.250 & -5.82 & 8.94 & 42.02 \\
0.5 & 1.625 & -6.43 & 13.76 & 53.12 \\
0.5 & 2.000 & 0.10 & 5.33 & 64.61 \\
\hline
\end{tabular}
\end{center}

\subsection{Energy Conservation}
Total energy increases with $g$, confirming stronger soliton interactions.

\section{Conclusion}
This study confirms that as nonlinearity increases, soliton interactions become more pronounced, leading to greater phase shifts and energy deviations. Future work will focus on exploring bound states in the system.

\end{document}
