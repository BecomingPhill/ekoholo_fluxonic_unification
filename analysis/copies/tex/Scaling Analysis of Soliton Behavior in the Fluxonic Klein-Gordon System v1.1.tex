\documentclass{article}
\usepackage{amsmath, listings}
\usepackage[margin=1in]{geometry}

\title{Scaling Analysis of Soliton Behavior in the Fluxonic Klein-Gordon System}
\author{Tshuutheni Emvula}
\date{February 20, 2025}

\begin{document}

\maketitle

\begin{abstract}
This paper analyzes soliton scaling in the nonlinear Klein-Gordon system within a fluxonic framework, hypothesizing that mass (\(m\)) and nonlinearity (\(g\)) variations influence gravitational interactions, testable via Bose-Einstein Condensate (BEC) modulation akin to gravitational shielding experiments. Simulations quantify phase shifts and energy conservation across \(m\) and \(g\), predicting measurable gravitational wave effects. These challenge General Relativity and quantum field theory, offering a deterministic gravitational model.
\end{abstract}

\section{Introduction}
Solitons in nonlinear systems offer insights into fundamental interactions (OCR Section 1). This study extends scaling analysis to a fluxonic context, aligning with OCR’s shielding paradigm (Section 3), linking \(m\) and \(g\) to testable gravitational effects.

\section{Hypothesis}
Soliton properties scale with:
\begin{itemize}
    \item \textbf{Mass (\(m\))}: Affects stability and gravitational coupling.
    \item \textbf{Nonlinearity (\(g\))}: Drives interaction strength, potentially measurable via wave attenuation (OCR Section 3).
\end{itemize}
Governed by:
\begin{equation}
\frac{\partial^2 \phi}{\partial t^2} - c^2 \frac{\partial^2 \phi}{\partial x^2} + m^2 \phi + g \phi^3 = 8 \pi G \rho,
\end{equation}
where \(\phi(x,t)\) is the fluxonic field, \(c = 1\), \(m\) and \(g\) vary, \(\rho\) is mass density (negligible here).

\section{Simulation Results and Observations}
Simulations analyze soliton collisions across \(m\) and \(g\):
\subsection{Phase Shift Dependence}
\begin{table}[h]
    \centering
    \begin{tabular}{|c|c|c|c|c|}
        \hline
        \(m\) & \(g\) & Phase Shift (Soliton 1) & Phase Shift (Soliton 2) & Final Energy \\
        \hline
        0.5 & 0.500 & 0.00 & 2.81 & 20.01 \\
        0.5 & 0.875 & -4.82 & 2.21 & 31.09 \\
        0.5 & 1.250 & -5.82 & 8.94 & 42.02 \\
        0.5 & 1.625 & -6.43 & 13.76 & 53.12 \\
        0.5 & 2.000 & 0.10 & 5.33 & 64.61 \\
        \hline
    \end{tabular}
    \caption{Scaling Analysis Results}
    \label{tab:scaling}
\end{table}
\subsection{Energy Conservation}
Energy increases with \(g\), indicating stronger interactions.

\section{Simulation Code}
\begin{lstlisting}[language=Python, caption=Soliton Scaling Simulation, label=lst:soliton]
import numpy as np
import matplotlib.pyplot as plt

# Parameters
L = 20.0
Nx = 200
dx = L / Nx
dt = 0.01
Nt = 500
c = 1.0
G = 1.0
rho = np.zeros(Nx)
params = [(0.5, 0.5), (0.5, 0.875), (0.5, 1.25), (0.5, 1.625), (0.5, 2.0)]

# Grid
x = np.linspace(-L/2, L/2, Nx)
results = []

for m, g in params:
    # Initial conditions: two solitons
    phi_initial = np.tanh((x + 5) / np.sqrt(2)) + np.tanh((x - 5) / np.sqrt(2))
    phi = phi_initial.copy()
    phi_old = phi.copy()
    phi_new = np.zeros_like(phi)

    # Time evolution
    for n in range(Nt):
        d2phi_dx2 = (np.roll(phi, -1) - 2 * phi + np.roll(phi, 1)) / dx**2  # Periodic boundaries
        phi_new = 2 * phi - phi_old + dt**2 * (c**2 * d2phi_dx2 - m**2 * phi - g * phi**3 + 8 * np.pi * G * rho)
        phi_old, phi = phi, phi_new

    # Phase shift (simplified peak analysis)
    peak1 = x[np.argmax(phi[:Nx//2])]
    peak2 = x[np.argmax(phi[Nx//2:]) + Nx//2]
    energy = np.sum(0.5 * ((phi - phi_old)/dt)**2 + 0.5 * (np.roll(phi, -1) - phi)/dx**2 + 0.5 * m**2 * phi**2 + 0.25 * g * phi**4)
    results.append((m, g, peak1 - (-5), peak2 - 5, energy))

# Plot for g = 2.0
plt.plot(x, phi_initial, label="Initial State")
plt.plot(x, phi, label="Final State (m=0.5, g=2.0)")
plt.xlabel("x")
plt.ylabel("φ(x,t)")
plt.title("Soliton Collision Simulation")
plt.legend()
plt.grid()
plt.show()
\end{lstlisting}

\section{Experimental Proposal}
Test via (OCR Section 3):
\begin{itemize}
    \item \textbf{Setup:} BEC with solitonic excitations (OCR Section 3.2).
    \item \textbf{Source:} Rotating mass (OCR Section 3.1).
    \item \textbf{Measurement:} LIGO interferometers (OCR Section 3.3) for wave shifts.
\end{itemize}

\section{Predicted Experimental Outcomes}
\begin{table}[h]
    \centering
    \begin{tabular}{|c|c|}
        \hline
        \textbf{Standard Prediction} & \textbf{Fluxonic Prediction} \\
        \hline
        Unaltered gravitational waves & Attenuation with \(g\) increase \\
        No soliton-gravity link & Phase shift-induced wave effects \\
        Fixed energy conservation & Energy scales with \(g\) \\
        \hline
    \end{tabular}
    \caption{Comparison of Predictions}
    \label{tab:predictions}
\end{table}

\section{Implications}
If confirmed (OCR Section 5):
\begin{itemize}
    \item Solitons influence gravity, challenging GR.
    \item Fluxonic framework unifies interactions.
    \item Engineering applications (OCR Section 5).
\end{itemize}

\section{Future Directions}
(OCR Section 6):
\begin{itemize}
    \item Explore bound states via simulations.
    \item Test higher \(m\), \(g\) values.
    \item Integrate with LIGO data analysis.
\end{itemize}

\end{document}