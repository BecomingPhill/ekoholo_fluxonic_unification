\documentclass{article}
\usepackage{amsmath, amssymb, graphicx}
\usepackage{hyperref}
\usepackage{listings}

\title{Soliton Collisions in the Nonlinear Klein-Gordon System}
\author{}
\date{}

\begin{document}
\maketitle

\section{Introduction}
This document outlines the numerical simulation of soliton collisions in the nonlinear Klein-Gordon equation with a $\phi^4$ potential. The primary objectives are to study soliton interactions, measure phase shifts, and validate energy conservation.

\section{Mathematical Framework}
The governing equation is:
\begin{equation}
\frac{\partial^2 \phi}{\partial t^2} - \frac{\partial^2 \phi}{\partial x^2} + m^2 \phi + g \phi^3 = 0
\end{equation}
where $m$ is the mass parameter and $g$ represents the nonlinear interaction.

We introduce two solitons moving toward each other:
\begin{equation}
\phi_1(x, 0) = \tanh\left(\frac{x - x_1}{\sqrt{2}}\right), \quad 
\phi_2(x, 0) = -\tanh\left(\frac{x - x_2}{\sqrt{2}}\right)
\end{equation}
with initial velocities $v_1$ and $v_2$.

\section{Numerical Implementation}
We employ a finite-difference scheme with absorbing boundaries. The initial conditions and discretization follow:
\begin{align}
\frac{\partial^2 \phi}{\partial t^2} &\approx \frac{\phi^{n+1}_i - 2\phi^n_i + \phi^{n-1}_i}{\Delta t^2}, \\
\frac{\partial^2 \phi}{\partial x^2} &\approx \frac{\phi^n_{i+1} - 2\phi^n_i + \phi^n_{i-1}}{\Delta x^2}.
\end{align}

\section{Results and Observations}
\subsection{Phase Shift Analysis}
The final positions of the solitons post-collision indicate a measurable phase shift:
\begin{itemize}
    \item Soliton 1 shifted by $6.53$ units.
    \item Soliton 2 shifted by $11.56$ units.
\end{itemize}
This confirms non-trivial interaction effects beyond simple superposition.

\subsection{Energy Conservation}
Total energy was analyzed over time:
\begin{itemize}
    \item Initial Energy: $47.56$
    \item Final Energy: $47.41$
    \item Energy Change: $0.32\%$
\end{itemize}
These results confirm numerical stability.

\section{Conclusion}
The simulation successfully models soliton collisions, demonstrating both phase shifts and energy conservation. Future work includes exploring bound states at different velocities and interaction strengths.

\end{document}
