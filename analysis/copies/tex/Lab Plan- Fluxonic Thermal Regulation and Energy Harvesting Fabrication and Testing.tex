\documentclass[a4paper,12pt]{article}
\usepackage[utf8]{inputenc}
\usepackage{amsmath}
\usepackage{listings}
\usepackage{geometry}
\geometry{margin=1in}

\title{Lab Plan: Fluxonic Thermal Regulation and Energy Harvesting Fabrication and Testing}
\author{Tshutheni Emvula}
\date{February 20, 2025}

\begin{document}

\maketitle

\section{Objective}
Fabricate and test a fluxonic thermal regulation material for directional heat flow and energy harvesting, based on the protocol in \emph{Fluxonic Thermal Regulation and Energy Harvesting: A Lab-Ready Experimental Guide}.

\section{Materials}
\begin{itemize}
    \item Thermoelectric substrate (e.g., bismuth telluride, graphene-based composites).
    \item Nano-patterning equipment (e.g., atomic layer deposition or sputtering).
    \item High-frequency electromagnetic field generator (0.1--10 THz).
    \item Temperature sensors (e.g., IR thermography, thermocouples).
    \item Electrical measurement tools (e.g., multimeter, oscilloscope).
\end{itemize}

\section{Experimental Synthesis Protocol}
\subsection{Material Composition}
\begin{itemize}
    \item Use a thermoelectric substrate such as bismuth telluride or graphene-based composites.
\end{itemize}

\subsection{Layered Structure}
\begin{itemize}
    \item Fabricate a layered composite with alternating fluxonic and conductive phases.
    \item Recommended layer thickness: 10--50 nm.
\end{itemize}

\subsection{Field Modulation}
\begin{itemize}
    \item Apply an oscillating electromagnetic field in the THz domain (0.1--10 THz) to align thermal wave interactions.
\end{itemize}

\section{Testing Procedure}
\begin{enumerate}
    \item Measure thermal asymmetry using temperature sensors (e.g., IR thermography) to confirm a temperature gradient of 5--15~$^\circ$C.
    \item Test energy harvesting efficiency by applying a heat gradient and THz field, measuring power output (expected 5--50 mW/cm$^2$) with a multimeter.
\end{enumerate}

\section{Simulation Support}
\subsection{Reproducible Code}
Below is the corrected Python code for simulating fluxonic thermal behavior, with OCR errors fixed (e.g., syntax in loop corrected).

\begin{lstlisting}[language=Python]
import numpy as np
import matplotlib.pyplot as plt

# Define spatial and temporal grid
Nx = 200  # Number of spatial points
Nt = 200  # Number of time steps
L = 10.0  # Spatial domain size
dx = L / Nx  # Spatial step size
dt = 0.01  # Time step

# Initialize spatial coordinates
x = np.linspace(-L / 2, L / 2, Nx)

# Define initial fluxonic temperature distribution
T = np.exp(-x ** 2) * np.cos(5 * np.pi * x)

# Interaction parameters
alpha = -0.3  # Heat flow control
beta = 0.2    # Nonlinearity for energy harvesting

# Initialize previous state
T_old = np.copy(T)
T_new = np.zeros_like(T)

# Time evolution loop
for n in range(Nt):
    d2T_dx2 = (np.roll(T, -1) - 2 * T + np.roll(T, 1)) / dx ** 2
    T_new = 2 * T - T_old + dt ** 2 * (d2T_dx2 + alpha * T + beta * T ** 3)
    T_old = np.copy(T)
    T = np.copy(T_new)

# Plot
plt.figure(figsize=(8, 5))
plt.plot(x, T, label="Fluxonic Thermal Regulation Field")
plt.xlabel("Position (x)")
plt.ylabel("Temperature Amplitude")
plt.title("Simulated Fluxonic Heat Flow & Energy Harvesting")
plt.legend()
plt.grid()
plt.show()
\end{lstlisting}

\section{Expected Outcomes}
\begin{itemize}
    \item Temperature gradient of 5--15~$^\circ$C indicating thermal diode behavior.
    \item Power output of 5--50 mW/cm$^2$ from waste heat conversion.
    \item Simulation output: Stable oscillatory temperature field.
\end{itemize}

\section{Notes}
\begin{itemize}
    \item Interpret "fluxonic phases" as needed (e.g., insulating layers, wave-engineered regions).
    \item Select a THz frequency (e.g., 1 THz) within the 0.1--10 THz range for testing.
\end{itemize}

\end{document}