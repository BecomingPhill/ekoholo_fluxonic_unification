\documentclass{article}
\usepackage{amsmath, amssymb, graphicx}

\title{Fluxonic Gravity and Emergent Black Holes: A Soliton-Based Approach}
\author{Independent Theoretical Study}
\date{\today}

\begin{document}

\maketitle

\begin{abstract}
This paper develops a fluxonic approach to gravity, demonstrating that black hole-like structures and gravitational effects can emerge from self-organizing solitonic fields. We derive a fluxonic gravity equation that replaces spacetime curvature with field stress-energy interactions and numerically simulate the collapse of fluxonic solitons. The results confirm the formation of horizon-like structures and the emission of Hawking-like radiation, challenging classical black hole thermodynamics and suggesting an alternative model for emergent gravity.
\end{abstract}

\section{Introduction}
General Relativity describes gravity as the curvature of spacetime. However, recent advancements in fluxonic physics suggest an alternative where gravity emerges from collective solitonic interactions rather than from spacetime deformation. We propose a model in which black holes arise naturally from fluxonic self-organization.

\section{Fluxonic Gravity Equation}
We formulate an alternative gravitational model based on the nonlinear fluxonic field equation:
\begin{equation}
    \frac{\partial^4 \phi}{\partial t^4} - \nabla^2 \frac{\partial^2 \phi}{\partial t^2} + m^2 \frac{\partial^2 \phi}{\partial t^2} + 3g\phi^2 \frac{\partial^2 \phi}{\partial t^2} + 6g\phi \left(\frac{\partial \phi}{\partial t} \right)^2 - \Lambda (g\phi^3 + m^2\phi + \Box \phi) = 0.
\end{equation}
This equation suggests that gravity is an emergent effect from fluxonic stress-energy variations rather than an intrinsic curvature of spacetime.

\section{Numerical Simulation of Fluxonic Black Hole Formation}
To validate our hypothesis, we simulate the collapse of fluxonic solitons and observe the formation of black hole-like structures. The key results include:
- The emergence of stable fluxonic black hole cores mimicking event horizons.
- The retention of gravitational energy without forming singularities.
- The gradual emission of energy over time, analogous to Hawking radiation.

\section{Fluxonic Hawking-Like Radiation}
We extend our simulation to analyze radiation emission from fluxonic black holes. The results show:
- A continuous outflow of energy from the fluxonic black hole boundary.
- A thermal radiation signature that mimics the properties of Hawking radiation.
- Evidence that fluxonic black holes dissipate energy without complete evaporation.

\section{Implications for Quantum Gravity}
The fluxonic gravity model challenges standard assumptions in gravitational physics:
1. **No Singularities:** The emergence of event horizons does not require spacetime singularities.
2. **Hawking-Like Radiation Without Quantum Fields:** Energy loss occurs naturally through fluxonic interactions.
3. **Potential Dark Matter Link:** Stable fluxonic gravitational structures may serve as an alternative explanation for dark matter.

\section{Conclusion}
Our findings demonstrate that black hole-like structures can form purely from fluxonic interactions, without invoking General Relativity’s spacetime curvature. These results open new avenues for understanding quantum gravity, dark matter, and astrophysical phenomena.

\end{document}

