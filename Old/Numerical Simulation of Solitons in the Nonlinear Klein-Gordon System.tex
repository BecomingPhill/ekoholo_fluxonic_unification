\documentclass{article}
\usepackage{amsmath, amssymb, graphicx}
\usepackage{hyperref}
\usepackage{listings}

\title{Numerical Simulation of Solitons in the Nonlinear Klein-Gordon System}
\author{}
\date{}

\begin{document}
\maketitle

\section{Introduction}
This document outlines the numerical implementation of soliton evolution in the nonlinear Klein-Gordon equation with a $\phi^4$ potential. The goal is to explore soliton stability, interactions, and scaling behaviors in the context of the Reciprocal System Theory.

\section{Mathematical Framework}
The nonlinear Klein-Gordon equation is given by:
\begin{equation}
\frac{\partial^2 \phi}{\partial t^2} - \frac{\partial^2 \phi}{\partial x^2} + m^2 \phi + g \phi^3 = 0
\end{equation}
where:
\begin{itemize}
    \item $\phi(x,t)$ is the scalar field.
    \item $m$ is a mass-like parameter.
    \item $g$ is the nonlinear interaction coefficient.
\end{itemize}

This equation supports \textbf{solitonic solutions} that remain stable due to the balance of dispersion and nonlinearity.

\section{Numerical Implementation}
We discretize the equation using finite differences:
\begin{align}
  \frac{\partial^2 \phi}{\partial t^2} &\approx \frac{\phi^{n+1}_i - 2\phi^n_i + \phi^{n-1}_i}{\Delta t^2}, \\
  \frac{\partial^2 \phi}{\partial x^2} &\approx \frac{\phi^n_{i+1} - 2\phi^n_i + \phi^n_{i-1}}{\Delta x^2}.
\end{align}

\textbf{Initial conditions:}
A standard kink soliton:
\begin{equation}
\phi(x, 0) = \tanh\left(\frac{x}{\sqrt{2}}\right)
\end{equation}
and an initial velocity perturbation:
\begin{equation}
\frac{\partial \phi}{\partial t} \bigg|_{t=0} = v \frac{d\phi}{dx}
\end{equation}

\textbf{Boundary conditions:} Absorbing boundaries are used to prevent artificial reflections.

\section{Python Implementation}
The numerical scheme is implemented in Python using finite-difference methods:
\begin{lstlisting}[language=Python]
import numpy as np
import matplotlib.pyplot as plt
import matplotlib.animation as animation

# Parameters
L = 20.0  # Spatial domain size
Nx = 200  # Number of spatial grid points
dx = L / Nx  # Spatial step size
dt = 0.01  # Time step size
Nt = 500  # Number of time steps
m = 1.0  # Mass parameter
g = 1.0  # Nonlinearity coefficient
v = 0.3  # Initial velocity of soliton

# Initialize spatial and temporal grids
x = np.linspace(-L/2, L/2, Nx)
phi = np.tanh(x / np.sqrt(2))  # Initial soliton profile
phi_old = np.tanh((x - v * dt) / np.sqrt(2))  # Apply initial velocity shift
phi_new = np.zeros_like(phi)

# Storage for visualization
phi_evolution = np.zeros((Nt, Nx))

# Time evolution using finite difference scheme
for n in range(Nt):
    d2_phi_dx2 = (np.roll(phi, -1) - 2 * phi + np.roll(phi, 1)) / dx**2
    phi_new = 2 * phi - phi_old + dt**2 * (d2_phi_dx2 - m**2 * phi - g * phi**3)
    phi_evolution[n, :] = phi
    phi_old = np.copy(phi)
    phi = np.copy(phi_new)

# Visualization
fig, ax = plt.subplots()
line, = ax.plot(x, phi_evolution[0, :], 'k')

def update(frame):
    line.set_ydata(phi_evolution[frame, :])
    ax.set_title(f"Time Step: {frame}")
    return line,

ani = animation.FuncAnimation(fig, update, frames=Nt, interval=30)
plt.xlabel("x")
plt.ylabel("φ(x,t)")
plt.title("Soliton Evolution in the Nonlinear Klein-Gordon System")
plt.show()
\end{lstlisting}

\section{Results and Observations}
\begin{itemize}
    \item The soliton remains \textbf{stable} throughout the simulation.
    \item The \textbf{initial velocity} $ v = 0.3 $ induces a slow rightward drift.
    \item The balance between dispersion and nonlinearity allows the soliton to retain its shape.
\end{itemize}

\section{Next Steps}
\begin{enumerate}
    \item \textbf{Soliton Collisions}: Introduce a second soliton with opposite velocity and observe interactions.
    \item \textbf{Phase Shift Analysis}: Measure displacement before and after collisions.
    \item \textbf{Energy Conservation}: Verify total energy remains constant.
    \item \textbf{Scaling Laws}: Explore how soliton properties depend on parameters $ m $ and $ g $.
\end{enumerate}

\end{document}

